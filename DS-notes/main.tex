\documentclass{book}
\usepackage[utf8]{inputenc}
\usepackage{geometry}
\usepackage{amsmath}
\usepackage{amssymb}
\usepackage{amsfonts}
\usepackage{graphicx}
\usepackage{tcolorbox}


\geometry{a4paper,
 total={170mm,257mm},
 left=15mm,
 right=15mm,
 top=10mm,}

\title{Cahier de kholle}

\begin{document}

\begin{center}
  \hrule
	\vspace{.4cm}
	{\textbf{\large Notes et reprises de DS/DM}}
\end{center}

{\textbf{Nom:}\ Arnaud Lelièvre \hspace{\fill} \vspace{0.5cm}}
{\textbf{}\  \hspace{\fill} \vspace{0.5cm}}
{\textbf{classe: MPSI 1}\ \hspace{\fill}}
\hrule
\date{}

\vspace{1cm}

\begin{center}
\textbf{\large Notes sur les DS et DM de l'année de sup à Pasteur en MPSI 1, et reprise de certaines questions}
\end{center}

\vspace{0.6cm}


\begin{center}
\textbf{\large Mathématiques}
\end{center} \vspace{0.2cm}



\section{DS du samedi 28 janvier}

\begin{tcolorbox}[width={14cm},colback={yellow!20!white},title={\textbf{Commentaire générale sur ce DS}},colbacktitle=red!40!white,coltitle=black]    
	Quelques manques sur le cours, et dans la justesse des utilisation des théorèmes, il faut CITER, ET CONNAITRE les hypothèses. \\
	De plus il faut plus d'attention sur ce qui est fait.
\end{tcolorbox}

\subsection{infos générales}

Note : 15.2 $\Leftrightarrow$ 7.8/20\\
Moyenne de classe : 20.8 $\Leftrightarrow$ 10/20 \\

Chapitres : dérivabilité $|$ structures algébriques \\ \\

\subsection{Erreurs :}

Erreurs "d'innatention" et erreurs de calcul \\ \\
$\rightarrow$ Une fonction a été dérivé puis intégré au lieu d'être dérivé 2 fois causans des termes qui ne s'annulent pas, l'ayant remarqué, j'ai noté avoir fait une erreur et marqué ce que j'étais censé obtenir, mais pas vu l'erreur en me relisant. \\ \\

Comment y remédier : \\ \\
$\rightarrow$ Prendre plus son temps sur les calculs compliqués même si c'est juste du calcul (qui est noramlement l'étape la plus simple du raisonnement) \\ \\
$\rightarrow$ Faire une meilleur relecture : c'est bien de voir la connerie, mais c'est mieux de la corriger ! \\ \\


Erreurs de précision : hypothèses importantes non données \\ \\
$\rightarrow$ Avant d'utiliser un théorème, il faut TOUJOURS justifier que l'on a toutes les hypothèses qui sont vérifié, MEME SI C'EST TRIVIALE ! De même pour la dérivabilité d'une fonction, on dit que c'est dérivable AVANT de dériver couillon ! \\ \\

Comment y remédier : \\ \\
$\rightarrow$ Faire les exercices avec plus de rigueure, même si ça implique en faire moins, les automatismes qui se formeront te fera gagner du temps net ! \\ \\
$\rightarrow$ Connais mieux ton cours bordèle, les hypothèses sont à savoir IMPERATIVEMENT ! \\ \\




\section{DM du lundi 6 mars (retour de vacances de février)}

\begin{tcolorbox}[width={14cm},colback={yellow!20!white},title={\textbf{Commentaire générale sur ce DM}},colbacktitle=red!40!white,coltitle=black]    

\end{tcolorbox}


\subsection{infos générales}

Note: Non noté \\
Moyenne de classe: Non noté \\

Chapitres : polynomes (arithmétique des polynomes et fraction rationnels des polynomes
) \\ \\

\subsection{Erreurs générales: } 

Montrer que des polynomes ont les memes racines ne suffit PAS pour dire que ce sont les mêmes polynomes ! \\ \\
Exemple: \\

$P(X) = (X-1)^2(X-2)$ et $Q(X) = (X-1)(X-2)^2$ ne sont pas les memes ! \\

\hspace{0.5cm} $\rightarrow$ Par contre: mêmes racines  + scindé à racines simples $\Rightarrow$ ascociés \\ \\


Quand on demande la STRUCTURE de $\mathbb{U}_n$, on attend: $\mathbb{U}_n$ est engendré par 1 seul élément $e^{ \frac{2i\pi}{n} }$ \\

"Propreté" de la copie: mettre plus en valeur les argument important: faire très attention sur ca sur les copies de DS ! \\

\subsection{Erreurs:}

\vspace{0.5cm}

Erreur de lecture: bien lire les questions ! \\ \\
$\rightarrow$ Ce n'est pas par ce que une question semble être un copié-collé du cours qu'il n'y a pas de subtilités \\ \\

Comment y remédier: \\ \\
$\rightarrow$ Analyser le sujet au début avec plus de rigueure \\ \\

Erreur de formulation, présentation: "$mq$" ne va pas à toutes les sauces ! \\ \\

Erreur de rédaction: pas de donc quand on résout une équation (sauf si ça en est vraiment un) \\ \\

Comment y remédier: \\ \\
$\rightarrow$ C'est le genre de truc facile à corriger en relecture ! \\ \\

Erreur de rédaction: Annoncer ce que l'on fait ! \\ \\





\section{DS du samedi 12 Mars}

\begin{tcolorbox}[width={14cm},colback={yellow!20!white},title={\textbf{Commentaire générale sur ce DM}},colbacktitle=red!40!white,coltitle=black]    

\end{tcolorbox}

\subsection{infos générales}

Note : 8.5 - partie A et B $|$ 6 partie C et D\\
Moyenne de classe : 9.2 - partie A et B $|$ 8.6 partie C et D \\

Chapitres : polynômes $|$ début de l'algèbre linéaire $|$ Dérivabilité $|$ Continuité \\ \\

\subsection{Erreurs :}

Erreurs "d'innatention" et erreurs de calcul \\ \\
$\rightarrow$ erreurs de calcul pur, il fait faire attention a bien faire, même le plus facile, et surtout le plus facile ! \\ \\

Erreurs de rédaction et précision\\ \\
$\rightarrow$ quand on fait une réccurence immédiate, il faut justifier un minimum ce qui nous permet de la faire, si c est demandé dans le sujet, c est qu'on appent pas "par reccurence immédiate" uniquement \\ \\

Que faire de mieux : \\ \\
$\rightarrow$ Bien analyser les question, chaques questions suivent le même model, quand une question à l'aire de se résoudre par analyse synthèse, commence par ça. \\ \\

A continuer : \\ \\
$\rightarrow$ ETRE METHODIQUE ET RIGOUREUX \\ \\

A faire : \\ \\
$\rightarrow$ etre rapide comme flash macqueen, sans pour autant perdre en rigueur $\rightarrow$ fais tes exercices soit très vite pour les simples (pas de temps a en faire des caisses quand tu sais déjà y répondre), mais surtout etre giga clean sur les exercices et question plus dure, histoire de ne pas perdre en rigueur (si tu sais le faire pour un truc difficile, ça devrait couler de source pour un bail simple)






\section{DM du lundi 20 mars }

\begin{tcolorbox}[width={14cm},colback={yellow!20!white},title={\textbf{Commentaire générale sur ce DM}},colbacktitle=red!40!white,coltitle=black]    
	Big erreurs de rédactions, et révise fort les DES en éléments simples et tt, même tout le chapitre sur les fractions des polynomes, c est fondamentale pour plein de trucs, donc à ne pas négliger pour les concours !
\end{tcolorbox}


\subsection{infos générales}

Note: 11.5/20\\
Moyenne de classe: ? \\

Chapitres : Espaces vectoriels \\ \\

\subsection{Erreurs générales: } 

Sur l'ex 1 : \\ \\
Pour la liberté de $(A^k B^{n-k})$, il ne fallait pas confondre la multiplicité des racines et le degré, et donc le "calque" de la démo sur la la liberté de polynomes de degré echelonnées. \\

$\rightarrow$ On pouvait en utilisant la multiplicié de dériver $k$ fois, puis évaluer P en $a$ et montrer que c'était 0 ! \\ \\


Sur l'ex 2 : \\ \\
Il était maladroit de dire qu'on fait une $analyse-synthese$, il fallait dire qu'on cherchait des $conditions$ $necesssaires$ ($CN$), il fallait en suite chercher simple (par exemple prendre une des fonction constante !) \\ \\


Sur l'ex 3 : \\ \\
Quand on écrit $\frac{P'}{P}$, il fait dire que l on prend $P \not= 0_{\mathbb{K}[X]}$ \\ \\
Ne pas oublier le coef dominant de $P$ ! \\ \\
Ne pas mettre d'unions dans les $EV$ \\ \\

\subsection{Erreurs : }

Rédaction : \\
$\rightarrow$ variables non déclarés, ce n'est même pas justifiable a ce stade de l'année \\
$\rightarrow$ plus JAMAIS de " $\therefore$ ", il déteste ça \\
$\rightarrow$ pas mettre de "car", il faut mettre ses arguments dans le bonne ordre putain ! \\
$\rightarrow$ Ecrire en toute lettres "$Analyse$" et "$synthese$" \\ \\

Compréhension : \\
$\rightarrow$ revoir les fractions rationelles




\section{DM du lundi 27 mars }

\begin{tcolorbox}[width={14cm},colback={yellow!20!white},title={\textbf{Commentaire générale sur ce DM}},colbacktitle=red!40!white,coltitle=black]    
	
\end{tcolorbox}


\subsection{infos générales}

Pour raccourcir des preuves : \\ \\
$\rightarrow$ ne pas négliger la notation " $Vect$ ", cela permet de "sauter" des preuves pour montrer que des choses sont des $\mathbb{K}$-$ev$ \\
$\rightarrow$ $Ker(\phi)$ est un sev ! \\
$\rightarrow$ penser aux \textbf{hyperplans} ! \\ \\

Question 6) \\ \\
$\rightarrow$ pour monter que $F \bigoplus G = E$, faire attention aux ordres des arguments, on ne montre PAS le caractère directe avant la somme !

\subsection{Erreurs :}

Erreurs de rédaction : \\ \\
$\rightarrow$ "$Vect$ d'éléments de E" ne se dit pas ! \\
$\rightarrow$ "Car il est $Ker(\phi)$ non plus ! \\ \\
 
Big erreur de compréhension : \\ \\
$\rightarrow$ Dans la question \textbf{3)}, quand on veut montrer que $G$ est stable par $u$ :

$u$ isomorphisme de $E$ dans $E$ $\not \Rightarrow$ $G$ stable par $u$ ! \\ \\

\hspace{2cm} Correction $[$Reprise$]$ : \\ \\
$\rightarrow$ On a $u =
\begin{pmatrix}
	2 & -1 & 2 \\
	2 & 2 & -1 \\
	-1 & 2 & 2	
\end{pmatrix}$ \\

\underline{Soit} $\begin{pmatrix} x \\ y \\ z \end{pmatrix} \in G$ \\
$\rhd$ mq $u \times \begin{pmatrix} x \\ y \\ z \end{pmatrix} \in G$ \\ \\
$\rightarrow$ $u \times \begin{pmatrix} x \\ y \\ z \end{pmatrix} \in G = x \begin{pmatrix} 2 \\ 2 \\ -1 \end{pmatrix} + y \begin{pmatrix} -1 \\ 2 \\ 2 \end{pmatrix} + z \begin{pmatrix} 2 \\ -1 \\ 2 \end{pmatrix}$




\section{DS du samedi 1er avril }

Note: 11.4\\
Moyenne de classe: 21 \\


\begin{tcolorbox}[width={14cm},colback={yellow!20!white},title={\textbf{Commentaire générale sur ce DS}},colbacktitle=red!40!white,coltitle=black]    
	Désastre, le DS a été raté de A à Z.
\end{tcolorbox}


\subsection{infos générales}

question 2) On a pas vraiment besoin de récurence \\
question 3) Ce n'était pas un copié-collé de l'exercice 4 du TD, plutot une adaptation \\
question 4) On avait $u^q= 0$, et $Ker(u^p) = Ker(u^{p+1}) = ... = ker(u^n)$ \\
question 6) bien utiliser les degrés échelonnés, bien dire que $D^n(P) \not = 0$ \\
question 8) Ne pas regarder $g^2$ o $D$, mais bien $g$ o $d = g$ o $g^2 - \lambda I_d$ \\
question 9) justifier quelles sont les seuls $sev$ stables par $D$ \\
question 10) utiliser la question 9 \underline{et} 8 \\
question 11) [ question raté pour beaucoup ] $\rhd \lambda > 0$ : $\triangle P = a \rightarrow g(P) = g(a) = ag(1)$, $\triangle g(1) = \mu \rightarrow g(1)^2 = g(g(1)) = g(\mu) = \mu g(1) = \mu^2 = g(1)^2$ \\
question 13) utiliser une base et la commutativité \\
question 14) erreur d'énnoncé \\
question 16) utiliser la question 3)


\subsection{erreurs}

$\rightarrow$ La question 3 a été faite en cours, (exercice 4), révise les exercices "fait en classe" bordèle, c'est le minimum, si t'es même pas capable de faire ca, viens pas au DS. \\ \\
$\rightarrow$ arrete d'essayer des arnaques ca ne sert à rien du tout hormis à énerver le correcteur \\ \\
$\rightarrow$ quand tu dis que c'est à degré échellonné, c'est pas dans les deux sens, c'est bien t'as vu les polynomes de degrés différents, dis qu'on a un système en changeant l'ordre des lignes, DIS EN TROP PLUTOT QUE PAS ASSEZ ! \\ \\
$\rightarrow$ quand on donne $g^2$ et qu'on demande des trucs sur $g$, faut essayer d'exprimer des trucs avec $g$, pas dire "ca marche sur $g^2$, donc ca deverait marcher pour $g$", encore une fois ARRETE TES ARNAQUES ! \\ \\
$\rightarrow$ c'est le CARDINALE pour une famille, et DIMENSION pour un $\mathbb{K}$-$ev$ connais mieux on cours, et utilise le bon language, c'est bien t'as vu le truc, mais putain appel le bien ! \\ \\
$\rightarrow$ le rest pue la grosse merde...




\section{DM du lundi 10 Avril}

\begin{tcolorbox}[width={14cm},colback={yellow!20!white},title={\textbf{Commentaire générale sur ce DM}},colbacktitle=red!40!white,coltitle=black]    
	En sah ça va à peu près, ATTENTION aux calcules, il faut BEAUCOUP plus expliquer ce qui est fait !
\end{tcolorbox}


\subsection{infos générales}

Note: 10/20\\
Moyenne de classe: ? \\

Chapitres : Probabilités \\ \\

\subsection{Erreurs générales: } 


\subsection{Erreurs : }

$\rightarrow$ Il faut beaucoup plus expliquer et "parler", notamment sur les graphes sur les arbres \\
$\rightarrow$ TOUJOURS mettre ce que l'on veut montrer (le $\blacktriangleright$) quand c'est pas directement la question \\
$\rightarrow$ Une somme de bernoulli est un binomiale ssi il y a indépendance (par le lemme des coallitions), il faut donc le dire ! \\
$\rightarrow$ Attention à l'ordre dans lequel tu déclares tes variables ! \\
$\rightarrow$ ne pas oublier de citer d'où viennent les inégalités




\section{DM du lundi 10 Avril}

\begin{tcolorbox}[width={14cm},colback={yellow!20!white},title={\textbf{Commentaire générale sur ce DM}},colbacktitle=red!40!white,coltitle=black]    
	En sah ça va à peu près, ATTENTION aux calcules, il faut BEAUCOUP plus expliquer ce qui est fait !
\end{tcolorbox}


\subsection{infos générales}

Note: 15/20\\
Moyenne de classe: ? \\

Chapitres : Probabiités | Matrices et $\mathcal{A} \mathcal{L}$  \\ \\

\subsection{Erreurs : }

$\rightarrow$ Ecrire les noms des théorèmes en entier \\
$\rightarrow$ commencer par les arguments, PAS DE "CAR" \\
$\rightarrow$ quand on a une limite, il faut justifier qu'elle existe \\






\section{DS du samedi avant les vacances de pâques}

\begin{tcolorbox}[width={14cm},colback={yellow!20!white},title={\textbf{Commentaire générale sur ce DS}},colbacktitle=red!40!white,coltitle=black]
	Pue la merde, arrete les remarques de merde et connais mieux ton cours, c est bien les remarques pour etre honnete, mais t es pas plus pénalisé si tu arnaque donc fuck it.
\end{tcolorbox}

\subsection{infos générales}

Note : 11.1 $\Leftrightarrow$ 5.9/20\\
Moyenne de classe : 20.4 $\Leftrightarrow$ 10/20 \\
infos : devoir bien passé (entre 25 et 40) $|$ cours su (entre 15 et 20) $|$ cours partiellement connu (entre 10 et 13) $|$ cours non su (inférieur à  10) \\ \\
Chapitres : algèbre linéaire $|$ probabilités $|$ dénombrement \\ \\

\subsection{Commentaires généraux :}

$\rightarrow$ De manière générale, on doit en faire plus \\
$\rightarrow$ être égal et semblable n'est pas la même chose ! \\
$\rightarrow$ une somme de Bernoulli n'est pas forcement une binomiale ! Il faut faire attentiona ux hypothèses \\
$\rightarrow$ attention à la structure du sujet, et aux hypothèses que le sujet met selon les parties \\

\subsection{Erreurs :}

$\rightarrow$ ne pas encadrer les résultats intermédiaires ! \\
$\rightarrow$ la famille posé n'est pas une base (je crois, j'avoue j'ai pas bien capté la correction là) \\
$\rightarrow$ justifier la concavité





\section{DM des vacances de pâques}

\begin{tcolorbox}[width={14cm},colback={yellow!20!white},title={\textbf{Commentaire générale sur ce DS}},colbacktitle=red!40!white,coltitle=black]
	Note plutot rude... Cependant, il faut faire attention à ne pas mélanger vecteur, matrices, endomorphismes ect... .
\end{tcolorbox}

\subsection{infos générales}

Note : 8/14 $\Leftrightarrow$ 11.43/20\\
Moyenne de classe : $\Leftrightarrow$ \\
Chapitres : algèbre linéaire \\ \\

\subsection{Erreurs :}

$\rightarrow$ bien utiliser les hypothèses, si on en a pas utilisé, y'a moyen qu'il y ait un soucis \\


\section{DS avec les copipes de Centrales}
\begin{tcolorbox}[width={14cm},colback={yellow!20!white},title={\textbf{Commentaire générale sur ce DS}},colbacktitle=red!40!white,coltitle=black]
	Connais mieux ton cours pour rouler sur ce qui est du cours ! Et quand un truc te semble bizarre, pars du principe que c'est toi qui as tors.
\end{tcolorbox}

\subsection{infos générales}

Note : 10.4/43 $\Leftrightarrow$ 6.1/20 \\
Moyenne de classe : 17/42 $\Leftrightarrow$ 10/20 \\
infos : devoir bien passé ($>$ 29) $|$ OK tiers (17 à 26) $|$ manque de soin dans les premières question (13 à 16) $|$ cours partiellement connu voire non su ($<$ 10) \\ \\
Chapitres : algèbre linéaire $|$ series \\ \\

\subsection{erreurs générales}

\textbf{\large{problème}} \\ \\
$\rightarrow$ bien dire qu'un idéal est stable par somme (je crois je l'ai di en sah) \\
$\rightarrow$ question 7) $-$ $(N_{e_1},...,N_{e_q})$ n'est pas forcement une base, c est une famille génératrice de $Im(N)$

attention à l'unicité (voir correction)

\underline{et} attention à ne pas confondre suplémentaire et complémentaire \\
$\rightarrow$ ne pas confondre les question 9) et 11) \\
$\rightarrow$ question 12) $-$ il ne fallait pas majorer $r$ par $n^2$, mais par $n$ \\ \\


\textbf{\large{exercice 1}} \\ \\
$\rightarrow$ attention au signes \\
$\rightarrow$ on additionne pas les équivalents \\


\subsection{erreurs}

$\rightarrow$ attention aux bornes de sommaion \\
$\rightarrow$ attention à l'indice des sommes (si on somme sur $k$, $i$, ou $n$) \\
$\rightarrow$ fiches tes cours, y'a des trucs comme les séries alternées qui sont du cours ptn ! \\
$\rightarrow$ on ne soustrait pas les équivalents ! \\




\section{DM du 21 mai}

\begin{tcolorbox}[width={14cm},colback={yellow!20!white},title={\textbf{Commentaire générale sur ce DS}},colbacktitle=red!40!white,coltitle=black]

\end{tcolorbox}

\subsection{infos générales}

Note : /20 \\
Chapitres : algèbre linéaire $|$ series \\ \\

\subsection{erreurs générales}

$\rightarrow$ Attention si $u_n$ est équivalent )  $(-n)^n v_n$, alors on pousse le DL plus loin ! \\
$\rightarrow$ Attention quand on separe un $ln(\frac{a}{b})$ en $ln(a) - ln(b)$, si $a$ et $b$ sont négatif, on a pas le droit \\


\section{DM du 5 Juin}

\begin{tcolorbox}[width={14cm},colback={yellow!20!white},title={\textbf{Commentaire générale sur ce DS}},colbacktitle=red!40!white,coltitle=black]
	Manque de rigueur et de précision et format sur les arguments
\end{tcolorbox}

\subsection{infos générales}

Note : 10/20 \\
Chapitres : intégrales $|$ algènre linéaire \\ \\

\subsection{erreurs générales}

$\rightarrow$ revoire les series de suites doublement récurrsives \\
$\rightarrow$ faire un vrai tableau pour la méthode DI \\
$\rightarrow$ préciser que c'est un segment pour les bornes atteintes




\begin{center}
\textbf{\large Physique}
\end{center}

\vspace{0.6cm}

\section{DS du 21 janvier 2022}

\begin{tcolorbox}[width={14cm},colback={yellow!20!white},title={\textbf{Commentaire générale sur ce DS}},colbacktitle=red!40!white,coltitle=black]    

\end{tcolorbox}

\subsection{infos générales}

note : 90/126 $\Leftrightarrow$ 14.29/20 \\
moyenne de la classe: /126 $\Leftrightarrow$ /20

\subsection{Erreurs :}

$\rightarrow$ attention aux calcules et au signes ! Il vaut mieux en faire moins mais faire 0 érreurs de calcule \\ \\
$\rightarrow$ pour n'analyse dimentionnel, pour $k_2$ par exemple, il faut pouvoir déterminer son unité avec d'autres formules dans lequelle il intervient : ici c'était l'expression de la force \\ \\
$\rightarrow$ oublie pas ta putain de constante d'intégration, c'est pas tout le temps 0 \\ \\
$\rightarrow$ bien réviser les équation adimentionnelles \\ \\



\section{DS du samedi 25 mars }

\begin{tcolorbox}[width={14cm},colback={yellow!20!white},title={\textbf{Commentaire générale sur ce DS}},colbacktitle=red!40!white,coltitle=black]    
	DS plutot "facile", mais que 47/132, des points sont à gagner sur la chimie ! Cependant un progrès dans le formalisme a été observé, mais la note reste vraiment décevante...
\end{tcolorbox}

\subsection{infos générales}

note : 47/132 $\Leftrightarrow$ 7.1/20 \\
moyenne de la classe: 51/132 $\Leftrightarrow$ 7.7/20


\subsection{Erreurs : }

$\rightarrow$ attention à ne pas oublier de vecteurs unitaires ! \\ \\
$\rightarrow$ Pour la partie sans approximation, il fait utiliser le $theoreme$ $de$ $l$'$energie$ $mecannique$ \\ \\
$\rightarrow$ Pour le solvant, il  fallait en prendre un qui ait une affinité \\ \\
$\rightarrow$ une équation quantitative est une réaction totale \\ \\



\begin{center}
\textbf{\large Option-info}
\end{center}

\vspace{0.6cm}

\section{DS du 14 janvier 2022}

\begin{tcolorbox}[width={14cm},colback={yellow!20!white},title={\textbf{Commentaire générale sur ce DS}},colbacktitle=red!40!white,coltitle=black]
	Beaucoup d'erreurs, il faut se la buter au $Caml$, si tu veux un jour mieux réussir, fais ses TD en entier tant que t es pas dans le top $30\%$, t est minorant si on compte pas les derniers qui pu la merde, c'est juste même pas consevable que tu fasse une bouse pareil une autre fois dans ta vie
\end{tcolorbox}

\subsection{infos générales}

note : 53/126 $\Leftrightarrow$ 8.41/20 \\
moyenne de la classe: 58.94/126 $\Leftrightarrow$ 9.35/20 \\ \\
$[$quasi minorant$]$

\subsection{Erreurs :}

$\rightarrow$ les signatures de fonctions ne sont tout simplement pas maîtrisés \\
$\rightarrow$ mets plein de parenthèses ! Quitte à ce que ce soit dégueu, mais correcte \\
$\rightarrow$ mets plus de commentares putain ! \\ \\
$\rightarrow$ fais gaffe à la compléxité putain, on est pas en NSI bordèle ! \\ \\
$\rightarrow$ Quand on fait un appel récursif, ont met tous les arguments ! \\




\section{DS du  2022}

\begin{tcolorbox}[width={14cm},colback={yellow!20!white},title={\textbf{Commentaire générale sur ce DS}},colbacktitle=red!40!white,coltitle=black]
	Tu n'as pas progressé par rapport au reste de la classe... . Tu fais partie des 3 connards qui commentent pas assez leur code, à changer !
\end{tcolorbox}

\subsection{infos générales}

note : 51/124 $\Leftrightarrow$ 8.23/20 \\
moyenne de la classe: 57.24/126 $\Leftrightarrow$ 9.23/20 \\ \\
$[$quasi minorant encore...$]$

\subsection{Erreurs :}

$\rightarrow$ Lis bien les putains de questions ! \\
$\rightarrow$ fais des fiches sur la syntaxe, pas seulement les notions \\


\section{DS du 3 juin 2023}

\begin{tcolorbox}[width={14cm},colback={yellow!20!white},title={\textbf{Commentaire générale sur ce DS}},colbacktitle=red!40!white,coltitle=black]
	Tu aurais pu gagner 8 places en étant plus soigneux et en faisant des commentaires ! pas fou mais pas degueu non plus
\end{tcolorbox}

\subsection{infos générales}

note : 61/145 $\Leftrightarrow$ 12.2/20 \\
moyenne de la classe: 67.47/145 $\Leftrightarrow$ 13.49/20 \\ \\

\subsection{Erreurs :}

$\rightarrow$ soin \\
$\rightarrow$ commentaires\\
$\rightarrow$ Ne pas oublier de cas (et surtout les cas d'initialisations) \\






\begin{center}
\textbf{\large Anglais}
\end{center}

\vspace{0.6cm}

\section{DS du samedi 7 janvier}

\begin{tcolorbox}[width={14cm},colback={yellow!20!white},title={\textbf{Commentaire générale sur ce DM}},colbacktitle=red!40!white,coltitle=black]    

\end{tcolorbox}

\subsection{infos générales}

Note : 11.6/20 \\

$\rightarrow$ text, intro et question : 4/8

$\rightarrow$ langue : 11/20

$\rightarrow$ méthode : 18/30 \\:


\subsection{Erreurs :}

$\rightarrow$ Attention à pas mélanger les mots

exemple : $-$ faith / fate \\ \\
$\rightarrow$ la restitution est parfois trop allusive (jsp trop ce qu'il veut dire par là) \\ \\
$\rightarrow$ ATTENTION à \underline{l'orthograph}, MEME DES MOTS FACILES \\ \\
$\rightarrow$ Des formulations ne sont pas bien !

exemple : $-$ les dates

\hspace{1.5 cm} $-$ des expressions \\

$\rightarrow$ ne pas commencer sa conclusion avec quelque chose de la forme "to sum up"

\subsection{Conseils généraux}

$\rightarrow$ Ne pas faire de jeux de mots dans l'intro et tout \\ \\
$\rightarrow$ \underline{souligner les sources} \\ \\
$\rightarrow$ essayer de nous trouver un "style" d'écriture \\ \\
$\rightarrow$ Ca ne sert à rien d'être giga précis sur les dates \\ \\
$\rightarrow$ les mois prennent des majuscules \\ \\
$\rightarrow$ NE PAS faire de mini résumés des articles et tout dans l'intro \\ \\



\section{DS du samedi 28 mai (environ)}

\begin{tcolorbox}[width={14cm},colback={yellow!20!white},title={\textbf{Commentaire générale sur ce DM}},colbacktitle=red!40!white,coltitle=black]    
	attention à bien cité les sources correctement, à l'orthographe, ne pas mettre de "we", on ne s'exprime pas $\leftarrow$ attention à la formulation, mieux organiser les arguments relevés
\end{tcolorbox}

\subsection{infos générales}

Note : 7.6/20 \\

$\rightarrow$ text, intro et question : 4/8

$\rightarrow$ langue : 7/20

$\rightarrow$ méthode : 12/30 \\:








\begin{center}
\textbf{\large Français}
\end{center}

\vspace{0.6cm}

\section{DS du samedi 7 janvier}

\begin{tcolorbox}[width={14cm},colback={yellow!20!white},title={\textbf{Commentaire générale sur ce Résumé}},colbacktitle=red!40!white,coltitle=black]    
	Manque de clarification trop collé au texte, il faut "traduire" le texte
\end{tcolorbox}

\subsection{infos générales}

Note : 7/20 \\

Moyenne de classe : 20.8 $\Leftrightarrow$ 10/20 \\

Type de DS : Résumé sur le texte de Jean Paul Sartre

voir commentaire générale


\end{document}
