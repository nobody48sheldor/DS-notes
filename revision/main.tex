\documentclass{book}
\usepackage[utf8]{inputenc}
\usepackage{geometry}
\usepackage{amsmath}
\usepackage{amssymb}
\usepackage{amsfonts}
\usepackage{graphicx}
\usepackage{tcolorbox}


\geometry{a4paper,
 total={170mm,257mm},
 left=15mm,
 right=15mm,
 top=10mm,}

\title{Cahier de kholle}

\begin{document}

\begin{center}
  \hrule
	\vspace{.4cm}
	{\textbf{\large Notes sur les révision de physique des vacances de Pâques}}
\end{center}

{\textbf{Nom:}\ Arnaud Lelièvre \hspace{\fill} \vspace{0.5cm}}
{\textbf{}\  \hspace{\fill} \vspace{0.5cm}}
{\textbf{classe: MPSI 1}\ \hspace{\fill}}
\hrule
\date{}

\vspace{1cm}

\begin{center}
\textbf{\large Notes sur les révision de physique des vacances de Pâques, pour le DS de physique sur tout}
\end{center}

\vspace{0.6cm}


\begin{center}
\textbf{\large Optique}
\end{center} \vspace{0.2cm}

\section{Optique - refléxion $|$ réfraction}

\underline{\Large{Cours :}} \\

$\rightarrow$ \underline{démo} de la refléxion totale

$\rightarrow$ \underline{démo} de la fibre optique \\ \\
\underline{\Large{Exercices :}} \\

$\rightarrow$ détécteur de pluie / truc de la première kholle

\hspace{1cm}$\rightarrow$ vraiment de la géométrie



\section{Optique - lentilles}

\underline{\Large{Cours :}} \\

$\rightarrow$ \underline{démo} image nette

$\rightarrow$ \underline{TP} revoir les TPs d'optique

$\rightarrow$ tracés \\ \\
\underline{\Large{Exercices :}} \\

$\rightarrow$ doublets de Huygens

\hspace{1cm}$\rightarrow$ faire $F'$ et $F$, car on sait pas ou est $O$




\begin{center}
\textbf{\large Eléctronique}
\end{center} \vspace{0.2cm}



\section{Eléctronique - RC + RLC}

\underline{\Large{Cours :}} \\

$\rightarrow$ normalement aquis \\ \\
\underline{\Large{Exercices :}} \\

$\rightarrow$ faire un peu de Thevenin-Norton

$\rightarrow$ faire un cricuit RC, LR et RLC \\



\section{Eléctronique impédences complexes et RSFE}

\underline{\Large{Cours :}} \\

$\rightarrow$ refaire les $\underline{z_C}$ et $\underline{z_L}$

$\rightarrow$ refaire les diviseurs de tension et courrant

$\rightarrow$ refaire les filtres en cascade

$\rightarrow$ fonctions de transfert et diagrammes de bode
$\rightarrow$ quadripôle


\section{Chimie}

\underline{\Large{Cours :}} \\

$\rightarrow$ étudier une réaction et une réaction double

$\rightarrow$ faire des configurations éléctroniques

$\rightarrow$ attention au voc et tt \\
\\
\underline{\Large{Exercices :}} \\

$\rightarrow$ \underline{ex 9 TD-C1} $-$ synthèse de l'amoniac $|$ en mode rapide \\

\end{document}
